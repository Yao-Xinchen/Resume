\documentclass[10pt, letterpaper]{article}

% Packages:
\usepackage[
    ignoreheadfoot, % set margins without considering header and footer
    top=2 cm, % seperation between body and page edge from the top
    bottom=1 cm, % seperation between body and page edge from the bottom
    left=2 cm, % seperation between body and page edge from the left
    right=2 cm, % seperation between body and page edge from the right
    footskip=1.0 cm, % seperation between body and footer
    % showframe % for debugging 
]{geometry} % for adjusting page geometry
\usepackage{titlesec} % for customizing section titles
\usepackage{tabularx} % for making tables with fixed width columns
\usepackage{array} % tabularx requires this
\usepackage[dvipsnames]{xcolor} % for coloring text
\definecolor{primaryColor}{RGB}{0, 79, 144} % define primary color
\usepackage{enumitem} % for customizing lists
\usepackage{fontawesome5} % for using icons
\usepackage{amsmath} % for math
\usepackage[
    pdftitle={Xinchen's CV},
    pdfauthor={Xinchen Yao},
    pdfcreator={LaTeX with RenderCV},
    colorlinks=false,
    urlbordercolor={0 0 0},
    pdfborderstyle={/S/U/W 0.5}
]{hyperref} % for links, metadata and bookmarks
\usepackage[pscoord]{eso-pic} % for floating text on the page
\usepackage{calc} % for calculating lengths
\usepackage{bookmark} % for bookmarks
\usepackage{lastpage} % for getting the total number of pages
\usepackage{changepage} % for one column entries (adjustwidth environment)
\usepackage{paracol} % for two and three column entries
\usepackage{ifthen} % for conditional statements
\usepackage{needspace} % for avoiding page brake right after the section title
\usepackage{iftex} % check if engine is pdflatex, xetex or luatex

% Ensure that generate pdf is machine readable/ATS parsable:
\ifPDFTeX
    \input{glyphtounicode}
    \pdfgentounicode=1
    \usepackage[T1]{fontenc}
    \usepackage[utf8]{inputenc}
    \usepackage{lmodern}
\fi

\usepackage{charter}

% Some settings:
\raggedright
\AtBeginEnvironment{adjustwidth}{\partopsep0pt} % remove space before adjustwidth environment
\pagestyle{empty} % no header or footer
\setcounter{secnumdepth}{0} % no section numbering
\setlength{\parindent}{0pt} % no indentation
\setlength{\topskip}{0pt} % no top skip
\setlength{\columnsep}{0.15cm} % set column seperation
\pagenumbering{gobble} % no page numbering

\titleformat{\section}{\needspace{4\baselineskip}\bfseries\large}{}{0pt}{}[\vspace{1pt}\titlerule]

\titlespacing{\section}{
    % left space:
    -1pt
}{
    % top space:
    0.3 cm
}{
    % bottom space:
    0.2 cm
} % section title spacing

\renewcommand\labelitemi{$\vcenter{\hbox{\small$\bullet$}}$} % custom bullet points
\newenvironment{highlights}{
    \begin{itemize}[
        topsep=0.10 cm,
        parsep=0.10 cm,
        partopsep=0pt,
        itemsep=0pt,
        leftmargin=0 cm + 10pt
    ]
}{
    \end{itemize}
} % new environment for highlights


\newenvironment{highlightsforbulletentries}{
    \begin{itemize}[
        topsep=0.10 cm,
        parsep=0.10 cm,
        partopsep=0pt,
        itemsep=0pt,
        leftmargin=10pt
    ]
}{
    \end{itemize}
} % new environment for highlights for bullet entries

\newenvironment{onecolentry}{
    \begin{adjustwidth}{
        0 cm + 0.00001 cm
    }{
        0 cm + 0.00001 cm
    }
}{
    \end{adjustwidth}
} % new environment for one column entries

\newenvironment{twocolentry}[2][]{
    \onecolentry
    \def\secondColumn{#2}
    \setcolumnwidth{\fill, 4.5 cm}
    \begin{paracol}{2}
}{
    \switchcolumn \raggedleft \secondColumn
    \end{paracol}
    \endonecolentry
} % new environment for two column entries

\newenvironment{twocolentrylong}[2][]{
    \onecolentry
    \def\secondColumn{#2}
    \setcolumnwidth{\fill, 6.0 cm}
    \begin{paracol}{2}
}{
    \switchcolumn \raggedleft \secondColumn
    \end{paracol}
    \endonecolentry
} % new environment for two column entries

\newenvironment{threecolentry}[3][]{
    \onecolentry
    \def\thirdColumn{#3}
    \setcolumnwidth{, \fill, 4.5 cm}
    \begin{paracol}{3}
    {\raggedright #2} \switchcolumn
}{
    \switchcolumn \raggedleft \thirdColumn
    \end{paracol}
    \endonecolentry
} % new environment for three column entries

\newenvironment{header}{
    \setlength{\topsep}{0pt}\par\kern\topsep\centering\linespread{1.5}
}{
    \par\kern\topsep
} % new environment for the header

% save the original href command in a new command:
\let\hrefWithoutArrow\href

% new command for external links:


\begin{document}
    \newcommand{\AND}{\unskip
        \cleaders\copy\ANDbox\hskip\wd\ANDbox
        \ignorespaces
    }
    \newsavebox\ANDbox
    \sbox\ANDbox{$|$}


    \begin{header}
        \fontsize{25 pt}{25 pt}\selectfont Xinchen Yao

        \vspace{5 pt}

        \normalsize
        \mbox{yao29@illinois.edu}%
        \kern 5.0 pt%
        \AND%
        \kern 5.0 pt%
        \mbox{xinchen.22@intl.zju.edu.cn}%
        \kern 5.0 pt%
        \AND%
        \kern 5.0 pt%
        \mbox{\href{https://yao-xinchen.github.io}{\textbf{yao-xinchen.github.io}}}%
    \end{header}

    \vspace{5 pt - 0.3 cm}

    \section{Education}

        Dual degree program in Zhejiang University - University of Illinois Urbana-Champaign Institute:

        \vspace{0.1cm}

        \begin{twocolentry}{
            Sep 2022 – May 2026
        }
            \textbf{University of Illinois Urbana-Champaign}, Urbana, Illinois, US\\
            BS in Computer Engineering, GPA: 3.77/4.0
        \end{twocolentry}

        \vspace{0.1cm}

        \begin{twocolentry}{
            Sep 2022 – May 2026
        }
            \textbf{Zhejiang University}, Zhejiang, China \\
            BS in Electronic and Computer Engineering, GPA: 3.83/4.3
        \end{twocolentry}

    \section{Technical Skills}

        \begin{onecolentry}
            \textbf{Programming:} Python, C/C++, CUDA, Matlab, Rust

            \vspace{0.10 cm}

            \textbf{Machine Learning:} PyTorch, Jax, Reinforcement Learning, Generative Models

            \vspace{0.10 cm}

            \textbf{Robotics:} Robot Simulators, Control Theory, ROS/ROS2, Computer Vision, SLAM

            \vspace{0.10 cm}

            \textbf{Systems:} Linux, Embedded Systems, Parallel Computing
        \end{onecolentry}

    \section{Awards and Honors}

        \begin{highlights}
            \item Outstanding Student in Academic Achievement, ZJU-UIUC Institute (2024 - 2025)
            \item Zhejiang University Scholarship (2024 - 2025)
            \item Second Prize in RoboMaster University League Competition (2023, 2024)
        \end{highlights}
    
    \section{Experience}

        \begin{twocolentry} {
            Undergraduate Researcher \\
            Jul 2025 - Present
        }
            \href{https://physicalintelligence-lab.github.io/}{\textbf{Physical Intelligence Lab}}, Zhejiang, China

            \vspace{0.05 cm}
            
            Zhejiang University - University of Illinois Urbana-Champaign Institute

            \vspace{0.05 cm}

            Advisor: \href{https://zjui.intl.zju.edu.cn/en/team/teacherinfo/2461}{\textbf{Prof. Hua Chen}} 
        \end{twocolentry}

        \vspace{0.10 cm}

        \begin{highlights}
            \item Developing locomotion controllers for bipedal robots and humanoids using reinforcement learning, achieving emergent behaviors that adapt to various terrains and conditions.
            \item Building evaluation pipelines to test trained robot policies across multiple simulation environments and deploying them to physical robots.
            \item Researching methods to reduce the performance gap between simulated and real-world robot operation through learned dynamics models.
            \item Improving reinforcement learning algorithms to achieve better training efficiency and performance stability.
        \end{highlights}

        \vspace{0.3 cm}

        \begin{twocolentry} {
            Undergraduate Researcher \\
            Sep 2024 - Jun 2025
        }
            \href{https://hdcl.mechanical.illinois.edu}{\textbf{Human Dynamics and Controls Lab}}, Illinois, US

            \vspace{0.05 cm}
            
            The Grainger College of Engineering, University of Illinois Urbana-Champaign

            \vspace{0.05 cm}

            Advisor: \href{https://bioengineering.illinois.edu/people/ethw}{\textbf{Prof. Elizabeth Hsiao-Wecksler}}

        \end{twocolentry}

        \vspace{0.10 cm}

        \begin{highlights}
            \item Developed a dynamics model for a self-balancing ballbot platform, simulating omniwheel behavior while maintaining computational efficiency.
            \item Implemented learning-based control to optimize balancing and movement performance and robustness.
            \item Integrated new force sensors into robot hardware, improving state estimation and control reliability.
        \end{highlights}
        
        \vspace{0.3 cm}

        \begin{twocolentry}{
            Control Group Leader \\
            Jun 2023 - Present
        }
            \href{https://github.com/Meta-Team}{\textbf{RoboMaster Meta Team}}, Zhejiang, China
            
            \vspace{0.05 cm}
            
            Zhejiang University - University of Illinois Urbana-Champaign Institute

            \vspace{0.05 cm}

            Advisor: \href{https://mechse.illinois.edu/people/profile/zjui-cui}{\textbf{Prof. Jiahuang Cui}}
            
        \end{twocolentry}

        \vspace{0.10 cm}

        \begin{highlights}
            \item Led control group to win second prize in RoboMaster regional competition, managing a team developing software for multiple competitive robots.
            \item Designed and implemented a distributed control system supporting real-time coordination for multiple robots with modular architecture.
            \item Developed software for motor control, sensor integration, autonomous navigation, and decision-making algorithms.
            \item Trained and mentored new team members on control theory, robot software development, and algorithm design.
            \item Collaborated with mechanical team to optimize robot design for improved controllability and performance.
        \end{highlights}
    
    \section{Projects}
    
        \begin{twocolentry}{
            Second Author
        }
            \textbf{Where to Learn: Analytical Policy Gradient Directed Exploration for On-Policy Robotic Reinforcement Learning}

        \end{twocolentry}

        \vspace{0.10 cm}

        \begin{highlights}
            \item Implemented a novel reinforcement learning algorithm that enhances exploration efficiency and training stability for robotic control.
            \item Evaluated the algorithm across multiple benchmark environments with comprehensive hyperparameter tuning and ablation studies.
            \item Successfully trained and deployed a walking policy on a physical biped robot for real-world validation.
            \item Website: \href{https://wheretolearn.github.io/}{wheretolearn.github.io}
        \end{highlights}

        \vspace{0.2 cm}

        \begin{twocolentry}{
            First Author
        }
            \textbf{Omni WBR: Learning Adaptive Hybrid Wheeled-Biped Robot Omnidirectional Locomotion via Position-Based Incentive}
        \end{twocolentry}

        \vspace{0.10 cm}
        \begin{highlights}
            \item Developed a reinforcement learning method to generate adaptive omnidirectional gaits for wheeled bipedal robots, enabling smooth movement on uneven terrains without explicit gait planning.
            \item Implemented a training and deploying pipeline to evaluate the method on physical robot.
            \item Demonstration available on my website: \href{https://yao-xinchen.github.io/projects/omni-wbr/}{yao-xinchen.github.io/projects/omni-wbr}
        \end{highlights}

        \vspace{0.2 cm}

        \begin{twocolentry}{
            Creator, Maintainer
        }
            \textbf{Meta-Team/Meta-ROS: Modular ROS2 Control Framework for multiple RoboMaster Competition Robots}
        \end{twocolentry}

        \vspace{0.10 cm}

        \begin{highlights}
            \item Created a comprehensive robot control system supporting multiple competitive robots with sensor integration, motion control, and autonomous coordination.
            \item Designed modular architecture enabling easy extension and real-time performance for competition scenarios.
            \item Integrated advanced capabilities including visual recognition, trajectory prediction, localization, and learning-based controllers.
            \item Code: \href{https://github.com/Meta-Team/Meta-ROS}{github.com/Meta-Team/Meta-ROS}
        \end{highlights}

        \vspace{0.2 cm}

        \begin{twocolentry}{
            Leader
        }
            \textbf{Custom Wheeled Biped Robot with RL-base Locomotion Controller}
        \end{twocolentry}

        \vspace{0.1 cm}

        \begin{highlights}
            \item Co-designed and built a custom wheeled biped robot with 4 degrees of freedom and closed chain structure.
            \item Trained a reinforcement learning-based locomotion controller and deployed it on the physical robot.
            \item Demonstration available on my website: \href{https://yao-xinchen.github.io/projects/wheeled-biped}{yao-xinchen.github.io/projects/wheeled-biped}
        \end{highlights}


    \section{Course Projects}

        \begin{twocolentrylong}{
            ECE391 Computer Systems
        }
            \textbf{Unix-Like RSIC-V Operating System, Including Kernel and Shell}
        \end{twocolentrylong}

        \begin{highlights}
            \item Implemented core operating system components including device drivers, filesystem, virtual memory, process management, and multitasking capabilities.
            \item Designed and implemented a Unix-style shell with process spawning and management.
            \item Composed a tutorial on workflow setup: \href{https://github.com/Yao-Xinchen/ECE391-Workflow}{github.com/Yao-Xinchen/ECE391-Workflow}
        \end{highlights}

        \vspace{0.1 cm}

        \begin{twocolentrylong}{
            ECE408 Applied Parallel Programming
        }
            \textbf{Convolution Neural Network Implementation in CUDA}
        \end{twocolentrylong}

        \begin{highlights}
            \item Implemented and optimized neural network inference using GPU parallel computing, achieving over 40\% performance improvement.
            \item Applied advanced optimization techniques including tensor cores, memory tiling, and specialized libraries for efficient computation.
        \end{highlights}

        \vspace{0.1 cm}

        \begin{twocolentrylong}{
            ECE484 Safe Autonomy
        }
            \textbf{Autonomous Diagonal and Parallel Parking}
        \end{twocolentrylong}

        \begin{highlights}
            \item Implemented localization and mapping system, and visual path recognition using deep learning.
            \item Designed and implemented autonomous parking system with diagonal and parallel parking maneuvers.
            \item Demonstration available on my website: \href{https://yao-xinchen.github.io/projects/auto-parking/}{yao-xinchen.github.io/projects/auto-parking}
        \end{highlights}
\end{document}