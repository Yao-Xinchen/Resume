\documentclass[10pt, letterpaper]{article}

% Packages:
\usepackage[
    ignoreheadfoot, % set margins without considering header and footer
    top=2 cm, % seperation between body and page edge from the top
    bottom=2 cm, % seperation between body and page edge from the bottom
    left=2 cm, % seperation between body and page edge from the left
    right=2 cm, % seperation between body and page edge from the right
    footskip=1.0 cm, % seperation between body and footer
    % showframe % for debugging 
]{geometry} % for adjusting page geometry
\usepackage{titlesec} % for customizing section titles
\usepackage{tabularx} % for making tables with fixed width columns
\usepackage{array} % tabularx requires this
\usepackage[dvipsnames]{xcolor} % for coloring text
\definecolor{primaryColor}{RGB}{0, 0, 0} % define primary color
\usepackage{enumitem} % for customizing lists
\usepackage{fontawesome5} % for using icons
\usepackage{amsmath} % for math
\usepackage[
    pdftitle={Xinchen's CV},
    pdfauthor={Xinchen Yao},
    pdfcreator={LaTeX with RenderCV},
    colorlinks=true,
    urlcolor=primaryColor
]{hyperref} % for links, metadata and bookmarks
\usepackage[pscoord]{eso-pic} % for floating text on the page
\usepackage{calc} % for calculating lengths
\usepackage{bookmark} % for bookmarks
\usepackage{lastpage} % for getting the total number of pages
\usepackage{changepage} % for one column entries (adjustwidth environment)
\usepackage{paracol} % for two and three column entries
\usepackage{ifthen} % for conditional statements
\usepackage{needspace} % for avoiding page brake right after the section title
\usepackage{iftex} % check if engine is pdflatex, xetex or luatex

% Ensure that generate pdf is machine readable/ATS parsable:
\ifPDFTeX
    \input{glyphtounicode}
    \pdfgentounicode=1
    \usepackage[T1]{fontenc}
    \usepackage[utf8]{inputenc}
    \usepackage{lmodern}
\fi

\usepackage{charter}

% Some settings:
\raggedright
\AtBeginEnvironment{adjustwidth}{\partopsep0pt} % remove space before adjustwidth environment
\pagestyle{empty} % no header or footer
\setcounter{secnumdepth}{0} % no section numbering
\setlength{\parindent}{0pt} % no indentation
\setlength{\topskip}{0pt} % no top skip
\setlength{\columnsep}{0.15cm} % set column seperation
\pagenumbering{gobble} % no page numbering

\titleformat{\section}{\needspace{4\baselineskip}\bfseries\large}{}{0pt}{}[\vspace{1pt}\titlerule]

\titlespacing{\section}{
    % left space:
    -1pt
}{
    % top space:
    0.3 cm
}{
    % bottom space:
    0.2 cm
} % section title spacing

\renewcommand\labelitemi{$\vcenter{\hbox{\small$\bullet$}}$} % custom bullet points
\newenvironment{highlights}{
    \begin{itemize}[
        topsep=0.10 cm,
        parsep=0.10 cm,
        partopsep=0pt,
        itemsep=0pt,
        leftmargin=0 cm + 10pt
    ]
}{
    \end{itemize}
} % new environment for highlights


\newenvironment{highlightsforbulletentries}{
    \begin{itemize}[
        topsep=0.10 cm,
        parsep=0.10 cm,
        partopsep=0pt,
        itemsep=0pt,
        leftmargin=10pt
    ]
}{
    \end{itemize}
} % new environment for highlights for bullet entries

\newenvironment{onecolentry}{
    \begin{adjustwidth}{
        0 cm + 0.00001 cm
    }{
        0 cm + 0.00001 cm
    }
}{
    \end{adjustwidth}
} % new environment for one column entries

\newenvironment{twocolentry}[2][]{
    \onecolentry
    \def\secondColumn{#2}
    \setcolumnwidth{\fill, 4.5 cm}
    \begin{paracol}{2}
}{
    \switchcolumn \raggedleft \secondColumn
    \end{paracol}
    \endonecolentry
} % new environment for two column entries

\newenvironment{threecolentry}[3][]{
    \onecolentry
    \def\thirdColumn{#3}
    \setcolumnwidth{, \fill, 4.5 cm}
    \begin{paracol}{3}
    {\raggedright #2} \switchcolumn
}{
    \switchcolumn \raggedleft \thirdColumn
    \end{paracol}
    \endonecolentry
} % new environment for three column entries

\newenvironment{header}{
    \setlength{\topsep}{0pt}\par\kern\topsep\centering\linespread{1.5}
}{
    \par\kern\topsep
} % new environment for the header

\newcommand{\placelastupdatedtext}{% \placetextbox{<horizontal pos>}{<vertical pos>}{<stuff>}
  \AddToShipoutPictureFG*{% Add <stuff> to current page foreground
    \put(
        \LenToUnit{\paperwidth-2 cm-0 cm+0.05cm},
        \LenToUnit{\paperheight-1.0 cm}
    ){\vtop{{\null}\makebox[0pt][c]{
        \small\color{gray}\textit{Last updated in September 2024}\hspace{\widthof{Last updated in September 2024}}
    }}}%
  }%
}%

% save the original href command in a new command:
\let\hrefWithoutArrow\href

% new command for external links:


\begin{document}
    \newcommand{\AND}{\unskip
        \cleaders\copy\ANDbox\hskip\wd\ANDbox
        \ignorespaces
    }
    \newsavebox\ANDbox
    \sbox\ANDbox{$|$}


    \begin{header}
        \fontsize{25 pt}{25 pt}\selectfont Xinchen Yao

        \vspace{5 pt}

        \normalsize
        \mbox{\hrefWithoutArrow{mailto:yao29@illinois.edu}{yao29@illinois.edu}}%
        \kern 5.0 pt%
        \AND%
        \kern 5.0 pt%
        \mbox{\hrefWithoutArrow{tel:+1-217-766-7971}{217 766 7971}}%
        \kern 5.0 pt%
        \AND%
        \kern 5.0 pt%
        \mbox{\hrefWithoutArrow{https://github.com/Yao-Xinchen}{github.com/Yao-Xinchen}}%
    \end{header}

    \vspace{5 pt - 0.3 cm}

    \section{Education}



        
        \begin{twocolentry}{
            Sept 2022 – May 2026
        }
            \textbf{University of illinois, Urbana Champaign}, BS in Computer Engineering\end{twocolentry}

        \vspace{0.10 cm}
        \begin{onecolentry}
            \begin{highlights}
                \item GPA: 3.77/4.0
            \end{highlights}
        \end{onecolentry}


    \section{Technical Skills}

        \begin{onecolentry}
            \textbf{Languages:} C/C++, Python, Rust, Matlab, Shell, RISC-V Assembly, CMake, XML, YAML \end{onecolentry}

        \vspace{0.2 cm}

        \begin{onecolentry}
            \textbf{Tools:} ROS2, PyTorch, Git, SSH, Isaac Gym, MoveIt, RViz, ONNX, STM32, Neovim \end{onecolentry}

        \vspace{0.2 cm}

        \begin{onecolentry}
            \textbf{Knowledge:} Reinforcement Learning, Control Theory, Motion Planning, Low-Level Communication Protocols \end{onecolentry}
    
    \section{Experience}



        
        \begin{twocolentry}{
            Jun 2023 - Sept 2024
        }
            \href{https://github.com/Meta-Team/Meta-ROS}{\textbf{Meta Team}}, Zhejiang, China
        
        \end{twocolentry}

        \vspace{0.10 cm}
        \begin{onecolentry}
            Advisor: \href{https://mechse.illinois.edu/people/profile/zjui-cui}{Jiahuang Cui}
            \begin{highlights}
                \item This a team in the competition \href{https://www.robomaster.com/en-US}{RoboMaster} held by DJI.
                \item Created an entire control system based on ROS2 for multiple robots.
                \item Integrated an auto-aiming program based on OpenCV.
                \item Designed a motion planning algorithm for a manipulator based on MoveIt.
                \item Helped with mechanical design and assembly.
            \end{highlights}
        \end{onecolentry}

        \vspace{0.2 cm}

        \begin{twocolentry} {
            Sept 2024 - Jun 2025
        }
            \href{https://hdcl.mechanical.illinois.edu}{\textbf{Human Dynamics Controls Lab}}, Illinois, US
        \end{twocolentry}

        \vspace{0.10 cm}
        \begin{onecolentry}
            Advisor: \href{https://bioengineering.illinois.edu/people/ethw}{Elizabeth Hsiao-Wecksler} 
            \begin{highlights}
                \item Built XACRO model for PURE Gen3.
                \item Control algorithm optimization for PURE Gen3.
            \end{highlights}
        \end{onecolentry}
        
        \vspace{0.2 cm}

    
    \section{Projects}
        
        \begin{twocolentry}{
            Meta-Team/Meta-ROS
        }
            \textbf{Control system based on ROS2}
        \end{twocolentry}

        \vspace{0.10 cm}
        \begin{onecolentry}
            \begin{highlights}
                \item An robot control system, including sensors, actuators, kinematics, decision, and manual control.
                \item Features: Supporting multiple robots, highly modular, dynamically configured.
                \item Tools used: C++, Python, ROS2 framework.
                \item \href{https://github.com/Meta-Team/Meta-ROS}{Code availability: github.com/Meta-Team/Meta-ROS}
            \end{highlights}
        \end{onecolentry}


        \vspace{0.2 cm}

    \section{Course Projects}

        \begin{twocolentry} {
            ECE398 FA24
        }
            \textbf{Robotics Project}, KIMLAB
        \end{twocolentry}

        \vspace{0.10 cm}
        \begin{onecolentry}
            Advisor: Joohyung Kim
            \begin{highlights}
                \item Built a duplicated Ringbot, including mechanics and hardware.
                \item Refactored the control framework with ROS2.
                \item Optimized the control algorithm with RL-based agent.
                \item Tools used: C++, Python, ROS2 framework, Isaac Gym, PyTorch, ONNX.
            \end{highlights}
        \end{onecolentry}

        \vspace{0.2 cm}

        \begin{twocolentry} {
            ECE391 FA24
        }
            \textbf{Computer Systems Engineering}
        \end{twocolentry}

        \vspace{0.10 cm}
        \begin{onecolentry}
            Advisor: Kirill Levchenko and Dong Kai Wang
            \begin{highlights}
                \item Implemented a Unix-like RISC-V OS with concurrency, vioblk and serial device, file system, virtual memory and system call.
                \item Wrote a shell in user space, supporting recursive calling.
            \end{highlights}
        \end{onecolentry}

        \vspace{0.2 cm}

\end{document}
