\documentclass[10pt, letterpaper]{article}

% Packages:
\usepackage[
    ignoreheadfoot, % set margins without considering header and footer
    top=2 cm, % seperation between body and page edge from the top
    bottom=1 cm, % seperation between body and page edge from the bottom
    left=2 cm, % seperation between body and page edge from the left
    right=2 cm, % seperation between body and page edge from the right
    footskip=1.0 cm, % seperation between body and footer
    % showframe % for debugging 
]{geometry} % for adjusting page geometry
\usepackage{titlesec} % for customizing section titles
\usepackage{tabularx} % for making tables with fixed width columns
\usepackage{array} % tabularx requires this
\usepackage[dvipsnames]{xcolor} % for coloring text
\definecolor{primaryColor}{RGB}{0, 0, 0} % define primary color
\usepackage{enumitem} % for customizing lists
\usepackage{fontawesome5} % for using icons
\usepackage{amsmath} % for math
\usepackage[
    pdftitle={Xinchen's CV},
    pdfauthor={Xinchen Yao},
    pdfcreator={LaTeX with RenderCV},
    colorlinks=true,
    urlcolor=primaryColor
]{hyperref} % for links, metadata and bookmarks
\usepackage[pscoord]{eso-pic} % for floating text on the page
\usepackage{calc} % for calculating lengths
\usepackage{bookmark} % for bookmarks
\usepackage{lastpage} % for getting the total number of pages
\usepackage{changepage} % for one column entries (adjustwidth environment)
\usepackage{paracol} % for two and three column entries
\usepackage{ifthen} % for conditional statements
\usepackage{needspace} % for avoiding page brake right after the section title
\usepackage{iftex} % check if engine is pdflatex, xetex or luatex

% Ensure that generate pdf is machine readable/ATS parsable:
\ifPDFTeX
    \input{glyphtounicode}
    \pdfgentounicode=1
    \usepackage[T1]{fontenc}
    \usepackage[utf8]{inputenc}
    \usepackage{lmodern}
\fi

\usepackage{charter}

% Some settings:
\raggedright
\AtBeginEnvironment{adjustwidth}{\partopsep0pt} % remove space before adjustwidth environment
\pagestyle{empty} % no header or footer
\setcounter{secnumdepth}{0} % no section numbering
\setlength{\parindent}{0pt} % no indentation
\setlength{\topskip}{0pt} % no top skip
\setlength{\columnsep}{0.15cm} % set column seperation
\pagenumbering{gobble} % no page numbering

\titleformat{\section}{\needspace{4\baselineskip}\bfseries\large}{}{0pt}{}[\vspace{1pt}\titlerule]

\titlespacing{\section}{
    % left space:
    -1pt
}{
    % top space:
    0.3 cm
}{
    % bottom space:
    0.2 cm
} % section title spacing

\renewcommand\labelitemi{$\vcenter{\hbox{\small$\bullet$}}$} % custom bullet points
\newenvironment{highlights}{
    \begin{itemize}[
        topsep=0.10 cm,
        parsep=0.10 cm,
        partopsep=0pt,
        itemsep=0pt,
        leftmargin=0 cm + 10pt
    ]
}{
    \end{itemize}
} % new environment for highlights


\newenvironment{highlightsforbulletentries}{
    \begin{itemize}[
        topsep=0.10 cm,
        parsep=0.10 cm,
        partopsep=0pt,
        itemsep=0pt,
        leftmargin=10pt
    ]
}{
    \end{itemize}
} % new environment for highlights for bullet entries

\newenvironment{onecolentry}{
    \begin{adjustwidth}{
        0 cm + 0.00001 cm
    }{
        0 cm + 0.00001 cm
    }
}{
    \end{adjustwidth}
} % new environment for one column entries

\newenvironment{twocolentry}[2][]{
    \onecolentry
    \def\secondColumn{#2}
    \setcolumnwidth{\fill, 4.5 cm}
    \begin{paracol}{2}
}{
    \switchcolumn \raggedleft \secondColumn
    \end{paracol}
    \endonecolentry
} % new environment for two column entries

\newenvironment{twocolentrylong}[2][]{
    \onecolentry
    \def\secondColumn{#2}
    \setcolumnwidth{\fill, 6.0 cm}
    \begin{paracol}{2}
}{
    \switchcolumn \raggedleft \secondColumn
    \end{paracol}
    \endonecolentry
} % new environment for two column entries

\newenvironment{threecolentry}[3][]{
    \onecolentry
    \def\thirdColumn{#3}
    \setcolumnwidth{, \fill, 4.5 cm}
    \begin{paracol}{3}
    {\raggedright #2} \switchcolumn
}{
    \switchcolumn \raggedleft \thirdColumn
    \end{paracol}
    \endonecolentry
} % new environment for three column entries

\newenvironment{header}{
    \setlength{\topsep}{0pt}\par\kern\topsep\centering\linespread{1.5}
}{
    \par\kern\topsep
} % new environment for the header

% save the original href command in a new command:
\let\hrefWithoutArrow\href

% new command for external links:


\begin{document}
    \newcommand{\AND}{\unskip
        \cleaders\copy\ANDbox\hskip\wd\ANDbox
        \ignorespaces
    }
    \newsavebox\ANDbox
    \sbox\ANDbox{$|$}


    \begin{header}
        \fontsize{25 pt}{25 pt}\selectfont Xinchen Yao

        \vspace{5 pt}

        \normalsize
        \mbox{\href{mailto:yao29@illinois.edu}{yao29@illinois.edu}}%
        \kern 5.0 pt%
        \AND%
        \kern 5.0 pt%
        \mbox{\href{mailto:xinchen.22@intl.zju.edu.cn}{xinchen.22@intl.zju.edu.cn}}%
        \kern 5.0 pt%
        \AND%
        \kern 5.0 pt%
        \mbox{\href{https://yao-xinchen.github.io}{\textbf{yao-xinchen.github.io}}}%
    \end{header}

    \vspace{5 pt - 0.3 cm}

    \section{Education}

        \begin{twocolentry}{
            Sep 2022 – May 2026
        }
            \textbf{Zhejiang University - University of Illinois Urbana-Champaign Institute}
        \end{twocolentry}

        \vspace{0.10 cm}

        Dual degree program:
        \begin{highlights}
            \item Zhejiang University: BS in Electrical and Computer Engineering
            \item University of Illinois Urbana-Champaign: BS in Computer Engineering
        \end{highlights}
        GPA: 3.77/4.0

    \section{Technical Skills}

        \begin{onecolentry}
            \textbf{Programming Languages:} Python, C/C++, CUDA, Matlab, Rust, RISC-V Assembly

            \vspace{0.10 cm}

            \textbf{Learning Frameworks:} PyTorch, RSL-RL, Jax, Brax

            \vspace{0.10 cm}

            \textbf{Simulators:} Isaac Lab, Isaac Gym, Genesis, Mujoco MJX, MJ Lab

            \vspace{0.10 cm}

            \textbf{Control Algorithms:} PID, LQR, MPC

            \vspace{0.10 cm}

            \textbf{Low-level Software:} ROS/ROS2, Concurrency, STM32, Communication Protocols, SLAM, OpenCV
        \end{onecolentry}
    
    \section{Experience}

        \begin{twocolentry} {
            Undergraduate Researcher \\
            July 2025 - Present
        }
            \href{https://physicalintelligence-lab.github.io/}{\textbf{Physical Intelligence Lab}}, Zhejiang, China

            \vspace{0.05 cm}
            
            Zhejiang University - University of Illinois Urbana-Champaign Institute

            \vspace{0.05 cm}

            Advisor: \href{https://zjui.intl.zju.edu.cn/en/team/teacherinfo/2461}{\textbf{Hua Chen}} 
        \end{twocolentry}

        \vspace{0.10 cm}

        \begin{highlights}
            \item Eliciting emergent behaviors in locomotion controllers for bipedal robots and humanoids, using Isaac Lab and Genesis simulation platforms with custom RSL-RL implementations.
            \item Developing sim-to-sim evaluation pipelines in Mujoco. Deploying trained policies onto various real robots.
            \item Research on minimizing sim-to-real gap with learned dynamics models and simulation alignment.
            \item Developing algorithms based on PPO and APG, which can achieve higher sample efficiency and stability.
        \end{highlights}

        \vspace{0.3 cm}

        \begin{twocolentry} {
            Undergraduate Researcher \\
            Sep 2024 - Jun 2025
        }
            \href{https://hdcl.mechanical.illinois.edu}{\textbf{Human Dynamics and Controls Lab}}, Illinois, US

            \vspace{0.05 cm}
            
            The Grainger College of Engineering, University of Illinois Urbana-Champaign

            \vspace{0.05 cm}

            Advisor: \href{https://bioengineering.illinois.edu/people/ethw}{\textbf{Elizabeth Hsiao-Wecksler}}

        \end{twocolentry}

        \vspace{0.10 cm}

        \begin{highlights}
            \item Developed a dynamics model in Genesis simulator for ballbot PURE Gen3 platform, which aims to accurately simulate the omniwheels while preserving computational efficiency.
            \item Implemented a learning-based control policies in Genesis to optimize balancing and movement performance and robustness for the PURE Gen3.
            \item Integrated new force sensors into PURE Gen3 hardware, improving state estimation and control reliability of existing model-based controller.
        \end{highlights}
        
        \vspace{0.3 cm}

        \begin{twocolentry}{
            Control Group Leader \\
            Jun 2023 - Present
        }
            \href{https://github.com/Meta-Team}{\textbf{RoboMaster Meta Team}}, Zhejiang, China
            
            \vspace{0.05 cm}
            
            Zhejiang University - University of Illinois Urbana-Champaign Institute

            \vspace{0.05 cm}

            Advisor: \href{https://mechse.illinois.edu/people/profile/zjui-cui}{\textbf{Jiahuang Cui}}
            
        \end{twocolentry}

        \vspace{0.10 cm}

        \begin{highlights}
            \item Won second prize in \href{https://www.robomaster.com/en-US}{RoboMaster} regional competitions, as the leader of the control group.
            \item Architected and implemented an ROS2-based distributed control system supporting multiple robots (sentry, hero, infantry, engineer robots) with modular asynchronous communication infrastructure.
            \item Developed software and hardware interfaces for real-time motor control and sensor data acquisition through their protocols, as well as high-level autonomous navigation and decision-making algorithms.
            \item Trained and mentored new team members on control theory fundamentals, ROS2 framework, robot hardware architecture, embedded systems programming, and algorithm design.
            \item Collaborated in mechanics-control co-design iterations to optimize robot kinematics, actuator selection, and sensor placement for improved controllability and performance.
        \end{highlights}
    
    \section{Projects}
    
        \begin{twocolentry}{
            Second Author
        }
            \textbf{Where to Learn: Analytical Policy Gradient Directed Exploration for On-Policy Robotic Reinforcement Learning}

        \end{twocolentry}

        \vspace{0.10 cm}

        \begin{highlights}
            \item A novel reinforcement learning algorithm combining Proximal Policy Optimization (PPO) with Adaptive Policy Gradient (APG) mechanisms to realize guided exploration in PPO, enhancing sample efficiency and stability.
            \item Implemented the algorithm in Brax, conducted training experiments in Mujoco MJX environments, performed hyperparameter tuning in benchmarks, and conducted ablation studies.
            \item Developed a new Mujoco MJX environment for our Tron1A biped robot, trained a policy with our new method, and deploy in reality for experiments.
            \item Website: \href{https://wheretolearn.github.io/}{\textbf{wheretolearn.github.io}}
        \end{highlights}

        \vspace{0.2 cm}

        \begin{twocolentry}{
            First Author
        }
            \textbf{Omni WBR: Learning Adaptive Hybrid Wheeled-Biped Robot Omnidirectional Locomotion via Position-Based Incentive}
        \end{twocolentry}

        \vspace{0.10 cm}
        \begin{highlights}
            \item An engineering method in reinforcement learning to generate emergent omnidirectional gaits for wheeled bipedal robots, enabling smooth adaption to uneven terrains without explicit gait planning.
            \item Designed the task setting to enable adaptive gaits, and implemented training environment in Isaac Lab for our Tron1A wheeled biped robot.
            \item Implement sim-to-real pipelines for the robot. Customized RSL-RL with concurrent teacher-student and applied Lipschitz constrain for better sim-to-real deployment.
            \item Demonstration available on my website: \href{https://yao-xinchen.github.io/projects/omni-wbr/}{\textbf{yao-xinchen.github.io/projects/omni-wbr/}}.
        \end{highlights}

        \vspace{0.2 cm}

        \begin{twocolentry}{
            Creator, Maintainer
        }
            \textbf{Meta-Team/Meta-ROS: Modular ROS2 Control Framework for multiple RoboMaster Competition Robots}
        \end{twocolentry}

        \vspace{0.10 cm}

        \begin{highlights}
            \item A comprehensive ROS2-based control system architecture encompassing sensor integration, actuator control, forward/inverse kinematics, and autonomous decision-making modules for multi-robot coordination.
            \item Supports heterogeneous robot platforms with dynamically configurable parameters, modular package structure for easy extension, and real-time performance optimization for competition scenarios.
            \item Includes high-level modules like visual recognition, trajectory prediction, lidar-based SLAM, and RL-based locomotion controllers.
            \item Code: \href{https://github.com/Meta-Team/Meta-ROS}{\textbf{github.com/Meta-Team/Meta-ROS}}
        \end{highlights}

        \vspace{0.2 cm}

        \begin{twocolentry}{
            Participant
        }
            \textbf{Custom Wheeled Biped Robot with RL-base Locomotion Controller}
        \end{twocolentry}

        \vspace{0.1 cm}

        \begin{highlights}
            \item Co-designed a custom Wheeled Biped Robot with 4 Degrees of Freedom and built a URDF model for it.
            \item Implemented a training environment in Isaac Gym, and trained a policy with custom RSL-RL.
            \item Developed a low-level control system for the robot with ROS2, and Deployed the trained policy onto it.
        \end{highlights}


    \section{Course Projects}

        \begin{twocolentrylong}{
            ECE391 Computer Systems
        }
            \textbf{Unix-Like RSIC-V Operating System, Including Kernel and Shell}
        \end{twocolentrylong}

        \begin{highlights}
            \item Implemented device drivers (UART, VirtIO block), filesystem I/O, virtual memory, process management, and preemptive multitasking with system calls for concurrent user program execution.
            \item Designed and Implemented a zsh-style interactive shell that can spawn child processes and restore interface upon completion.
        \end{highlights}

        \vspace{0.1 cm}

        \begin{twocolentrylong}{
            ECE408 Applied Parallel Programming
        }
            \textbf{Convolution Neural Network Implementation in CUDA}
        \end{twocolentrylong}

        \begin{highlights}
            \item Implemented and optimized CUDA-based CNN inference, achieving over 40\% performance improvement.
            \item Applied advanced techniques including tensor cores, CUDA stream, Joint Register and Shared Memory Tiling, and cuBLAS for higher memory efficiency.
        \end{highlights}

        \vspace{0.1 cm}

        \begin{twocolentrylong}{
            ECE484 Safe Autonomy
        }
            \textbf{Autonomous Diagonal and Parallel Parking}
        \end{twocolentrylong}

        \begin{highlights}
            \item Implemented Simultaneous Localization and Mapping (SLAM) with Fast-LIO, and path recognition with ENet.
            \item Designed and implemented a autonomous parking system with predefined diagonal and parallel maneuvers.
            \item Demonstration available on my website: \href{https://yao-xinchen.github.io/projects/auto-parking/}{\textbf{yao-xinchen.github.io/projects/auto-parking}}
        \end{highlights}
\end{document}